\documentclass{homework}

\title{Your Title}
\author{Y.\,O. Urname}

\begin{document}

\maketitle

\exercise
Exercises are automatically numbered, starting from one. Essential packages such as \texttt{amsmath} and \texttt{hyperref} are included by default.

Instead of being indented, paragraphs are separated by some white space.

\exercise*
Each exercise (except the first) starts on a new page. You can disable this behavior using the starred version of the command: \verb|\exercise*|.

Now, let's consider a mathematical example.

\begin{definition}
    The \emph{standard inner product} of two vectors $\vec x, \vec y \in \R^n$ is defined as
    \[
        \vec x * \vec y \coloneqq x_1 y_1 + \dots + x_n y_n.
    \]
\end{definition}

Next to definitions, environments for theorems and lemmas are included as well. Furthermore, you can easily define your own with the \verb|\NewTheorem| command.

Note that \texttt{*} can be used instead of \verb|\cdot|, and \verb|\R| instead of \verb|\mathbb{R}|. (For a normal asterisk, use \verb|\ast|.) Of course, there are also macros for the natural numbers etc. Commands such as \verb|\abs{}| and \verb|\set{}| can be used to create (scaled) delimiters. For example,
\[
    \abs{\frac{1}{1 - \lambda h}} \le 1
    \qquad\text{and}\qquad
    \bigcup_{i=1}^n \; \set{z \in \C \mid \abs*{z - a_{ii}} \le {\sum\nolimits_{j \ne i}} \abs*{a_{ij}}}.
\]
The starred version of these commands disables the auto-scaling.

\exercise[Rec--2.1]
Optionally, you can fully customize the numbering of each exercise \dots

\setcounter{exercise}{7}
\exercise*
\dots{} or skip a few, using the \verb|\setcounter{exercise}{x}| command.

For more information, refer to \url{https://github.com/gijs-pennings/latex-homework}.

\end{document}
